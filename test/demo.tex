\documentclass{article}

\usepackage{amsmath}
\usepackage{amscd}
\usepackage[tableposition=top]{caption}
\usepackage{ifthen}
\usepackage[utf8]{inputenc}

\usepackage{Sweave}
\begin{document}

\title{mmod Demo}
\author{David Winter}
\maketitle

This is a short demo of \verb@mmod@ command in R. 

As an example, we are going to examine the \verb@nancycats@ data that comes
with \verb@adegenet@. This dataset contains microsattelite genotypes
taken from feral cats in Nancy, France. Our goal is to see (a) to what degree
populations of these cats are differentiated from each other and (b) if that
differentiation can be explained by the geographic distance between subpopulations. 

So let's start. 

\begin{Schunk}
\begin{Sinput}
> library(mmod)
> data(nancycats)
> nancycats
\end{Sinput}
\begin{Soutput}
   #####################
   ### Genind object ### 
   #####################
- genotypes of individuals - 

S4 class:  genind
@call: genind(tab = truenames(nancycats)$tab, pop = truenames(nancycats)$pop)

@tab:  237 x 108 matrix of genotypes

@ind.names: vector of  237 individual names
@loc.names: vector of  9 locus names
@loc.nall: number of alleles per locus
@loc.fac: locus factor for the  108 columns of @tab
@all.names: list of  9 components yielding allele names for each locus
@ploidy:  2
@type:  codom

Optionnal contents: 
@pop:  factor giving the population of each individual
@pop.names:  factor giving the population of each individual

@other: a list containing: xy 
\end{Soutput}
\end{Schunk}

The functions in \verb@mmod@ work on genind objects, which are provided by
 \verb@adegenet@, you can read your data in using \verb@read.genepop()@) to 
 read data in form genepop files. 
 
Once we have the data to work on, we want to know how strong any divergence 
between sub-populations might be. 

\begin{Schunk}
\begin{Sinput}
> diff_stats(nancycats)
\end{Sinput}
\begin{Soutput}
$per.locus
             Hs        Ht        Gst Gprime_st          D
fca8  0.7708277 0.8614311 0.10517782 0.4810570 0.42006021
fca23 0.7415102 0.7992621 0.07225650 0.2924881 0.23738411
fca43 0.7416796 0.7935120 0.06532017 0.2645865 0.21319208
fca45 0.7273320 0.7641204 0.04814486 0.1845960 0.14335289
fca77 0.7766369 0.8655618 0.10273670 0.4822798 0.42300076
fca78 0.6316202 0.6772045 0.06731245 0.1899390 0.13147655
fca90 0.7369587 0.8141591 0.09482221 0.3770880 0.31183460
fca96 0.6699736 0.7654561 0.12473941 0.3937947 0.30740024
fca37 0.5623259 0.6024354 0.06657894 0.1574662 0.09737005

$global
        Hs         Ht    Gst_est  Gprime_st      D_het     D_mean 
0.70654052 0.77146027 0.08415178 0.29942062 0.23504860 0.20017978 
\end{Soutput}
\end{Schunk}

OK, what's all that then? \verb@Hs@ and \verb@Ht@ are measures of the 
heterozgosity expected for this population with and without 

